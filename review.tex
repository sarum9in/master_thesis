\input head
\renewcommand{\baselinestretch}{1.3} \setlength{\lineskiplimit}{-10cm}

\begin{document}

\begin{titlepage}

\thispagestyle{empty}

\begin{center}
    \large \textbf{\uppercase{Рецензия}}

    \vspace{1em}

    На диссертацию Филиппова А.Н. <<Модели RPC в проектировании олимпиадного
    сервера>> представленную на соискание степени магистра по направлению
    09.04.04-1 \\*
    «Информатика и вычислительная техника»
\end{center}

\vspace{2em}

Диссертация Филиппова А.Н. состоит из введения, 3-х глав, заключения
и приложения. Во введении показана актуальность темы диссертации, изложен объект
исследования, предмет исследования и цель работы, перечислены научно-технические
задачи, необходимые для решения поставленной цели и представлены результаты
работы, выносимые на защиту.

В первой главе выполнен обзор существующих методов межпроцессных взаимодействий,
приведена классификация систем вызова удалённых процедур и обоснованы задачи,
сформулированные во введении.

Вторая глава посвящена разработке моделей вызовов удалённых процедур
в олимпиадном сервере, а также разработке протокола асинхронного вызова
удалённых процедур.

В третьей главе рассмотрена реализация протокола RPC, разработанного во второй
главе, представлено применение полученной реализации в олимпиадном сервере.

В заключении диссертационной работы сформулированы основные выводы и результаты
выполнения работы.

В приложении приведены тексты разработанного программного обеспечения.

Структура и предложенные решения обоснованы, выполнены с технической
и научной точки зрения грамотно, и правильность предложенных решений
подтверждена экспериментально.

%
Тем не менее, работа имеет недостаток. \textbf{Предлагаемый способ предварительной обработки обладает меньшим быстродействием по сравнению с использовавшимся до этого.}
%
%Однако работа имеет ряд недостатков.
%Не смотря на обоснование использования инфраструктуры открытых ключей,
%в представленной работе упущено описание альтернативной модели управления
%цифровыми сертификатами — Web of Trust.

Несмотря на указанный недостаток, диссертационная работа Филиппова А.Н. является
завершенным научным исследованием, выполненным на актуальную тему.
Положения, вынесенные автором на защиту, обладают научной новизной
и практической ценностью. Рассмотренная работа соответствует требованиям,
предъявляемым к магистерским диссертациям, и заслуживает оценки "отлично",
а ее автор — присуждения степени магистра.

\vfill
\noindent Рецензент \\*
д.ф. - м.н. \hfill М.М. Горохов

\end{titlepage}

\end{document}

% vim: spell spelllang=ru
