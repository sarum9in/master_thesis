%================
% Разметка страницы
\usepackage{fancyhdr}
\fancypagestyle{plain}{
\fancyhf{}
\fancyhead[R]{\thepage}
\renewcommand{\headrulewidth}{0pt}
\renewcommand{\footrulewidth}{0pt}
}

%================
% Остальное
\renewcommand{\baselinestretch}{1.5}
\frenchspacing
\newcommand{\myenquote}[1]{``#1''}

%================
% Формат списков
\renewcommand{\labelenumi}{\arabic{enumi})}
\renewcommand{\labelenumii}{\arabic{enumi}.\arabic{enumii})}
\renewcommand{\labelenumiii}{\labelenumii\arabic{enumiii})}
\renewcommand\textbullet{--}

%================
% Формат области печати
\usepackage[a4paper]{geometry}
  \geometry{left=3cm}
  \geometry{right=1cm}
  \geometry{top=2cm}
  \geometry{bottom=2cm}

%================
% Формат заголовков
\usepackage[explicit]{titlesec}

\AtBeginDocument{
\renewcommand{\contentsname}{Содержание}
\renewcommand{\bibname}{Список литературы}
}
\renewcommand{\appendixname}{Приложение}

\renewcommand{\thefigure}{\arabic{chapter}.\arabic{figure}}
\renewcommand{\theequation}{\arabic{equation}}

\setcounter{secnumdepth}{4}
%\let\appendix\chapter
\titleformat{\chapter}[hang]
    {\normalfont\center}
    {\thechapter.}
    {1em}
    {\uppercase{#1}}
\titlespacing{\chapter}
    {0cm} %left
    {-2em} %before
    {1cm} %after

\titleformat{\section}[hang]
    {\normalfont}
    {\thesection.}
    {1em}{#1}
\titlespacing{\section}
    {1.5cm} %left
    {1.5em} %before
    {0.5cm} %after

\titleformat{\subsection}[hang]
    {\normalfont}
    {\thesubsection.}
    {1em}{#1}
\titlespacing{\subsection}
    {1.5cm} %left
    {1.5em} %before
    {0.5cm} %after

\titleformat{\subsubsection}[hang]
    {\normalfont}
    {\thesubsubsection.}
    {1em}{#1}
\titlespacing{\subsubsection}
    {1.5cm} %left
    {1em} %before
    {1em} %after

\titleformat{\paragraph}[runin]
    {\normalfont}
    {\theparagraph.}
    {1em}{#1}
\titlespacing{\paragraph}
    {1.5cm} %left
    {1em} %before
    {1ex} %after

%==================
% Оглавление
\usepackage{tocloft}
\usepackage{etoolbox}

\renewcommand{\cftchapaftersnum}{.}
\renewcommand{\cftsecaftersnum}{.}
\renewcommand{\cftsubsecaftersnum}{.}
\renewcommand{\cftsubsubsecaftersnum}{.}
\renewcommand{\cftparaaftersnum}{.}

\renewcommand{\cftchappagefont}{\mdseries}
\renewcommand{\cftpartleader}{\cftdotfill{\cftdotsep}}
\renewcommand{\cftchapleader}{\cftdotfill{\cftdotsep}}

\makeatletter
\patchcmd{\l@chapter}
    {\cftchapfont #1}   % search pattern
    {\cftchapfont {#1}} % replace by
    {}                  % success
    {}                  % failure
\makeatother
\renewcommand\cftchapfont{\mdseries\MakeUppercase}

\makeatletter
\patchcmd{\@cftmaketoctitle}
    {\cfttoctitlefont\contentsname}
    {\cfttoctitlefont{\contentsname}}
    {}
    {}
\makeatother
\renewcommand{\cfttoctitlefont}[1]{\large\centering\MakeUppercase{#1}\par}

\setlength{\cftbeforechapskip}{0.5em}
\setlength{\cftbeforesecskip}{0.5em}
\setlength{\cftbeforesubsecskip}{0.5em}
\setlength{\cftbeforetoctitleskip}{-2em}

%==================
% Списки
\usepackage{enumitem}
\setlist{nolistsep}
\setenumerate[0]{leftmargin=2cm}
\setenumerate[1]{leftmargin=2cm}
\setenumerate[2]{leftmargin=1cm}
\setitemize[0]{leftmargin=2cm}

%==================
% Код
\usepackage{listings}
\lstset{ %
language=C++,                   % the language of the code
basicstyle=\ttfamily\footnotesize,% the size of the fonts that are used for the code
numbers=left,                   % where to put the line-numbers
numberstyle=\footnotesize,      % the size of the fonts that are used for the line-numbers
stepnumber=2,                   % the step between two line-numbers. If it's 1, each line
                                % will be numbered
numbersep=5pt,                  % how far the line-numbers are from the code
%backgroundcolor=\color{white},  % choose the background color. You must add \usepackage{color}
showspaces=false,               % show spaces adding particular underscores
showstringspaces=false,         % underline spaces within strings
showtabs=false,                 % show tabs within strings adding particular underscores
frame=single,                   % adds a frame around the code
tabsize=2,                      % sets default tabsize to 2 spaces
captionpos=b,                   % sets the caption-position to bottom
breaklines=true,                % sets automatic line breaking
breakatwhitespace=false,        % sets if automatic breaks should only happen at whitespace
title=\lstname,                 % show the filename of files included with \lstinputlisting;
                                % also try caption instead of title
escapeinside={\%*}{*)},         % if you want to add a comment within your code
morekeywords={*,...}            % if you want to add more keywords to the set
}
