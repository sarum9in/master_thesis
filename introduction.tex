\chapter*{Введение}
\addcontentsline{toc}{chapter}{Введение}
Разработка систем автоматизации образовательных процессов является важным этапом
информатизации общества. Это позволяет повысить эффективность образовательного
процесса путём снижения нагрузки на сотрудников, исключения ошибок и
неточностей, допускаемых человеком. Также это позволяет создавать программы
удалённого обучения без непосредственного участия преподавателя.

Автоматизация в сфере образования затрагивает множество аспектов,
от электронных табелей успеваемости и расписаний занятий до автоматизированных
тестирований учащихся.

Неотъемлемой частью образовательного процесса является практическая работа.
Она характеризуется решением определённого класса задач в пределах изучаемого
курса. Существуют группы задач, решения для которых могут быть проверены
в автоматическом режиме, к примеру из курсов программной инженерии.
Существуют системы, автоматизирующие проверку решений для таких задач,
а также проведения лабораторных и практических работ, соревнований и олимпиад.

С ростом количества пользователей таких систем возрастают и системные требования.
Одним из способов решения проблем производительности является
распределение нагрузки между отдельными серверами проверяющей системы.
Это позволяет добиться горизонтальной масштабируемости, то есть возможности
увеличения мощности системы путём увеличения количества узлов сети.
Такая система уже будет являться распределённой \cite{distributed}.
Для них характерно распределение функций и ресурсов между множеством узлов.
Такие системы часто реализуют избыточность ресурсов, что позволяет им
оставаться работоспособными даже при выходе части узлов из строя.

Одной из проблем, которые необходимо решить при создании такой системы,
это координация работы узлов системы и передача результатов вычислений
между ними. Для решения этой проблемы может применяться концепция RPC --
remote procedure call. Эта концепция позволяет организовать процесс передачи
данных в виде запрос-ответ. Различные типы запросов и передаваемых данных
создают необходимость использования различных технологий RPC.

\textbf{Целью работы} является анализ различных моделей использования RPC
и разработка технологии RPC, которая требуется олимпиадной системе, но не имеет
аналогов, удовлетворяющих требованиям.

Для достижения цели необходимо решить следующие \textbf{задачи}:
\begin{itemize}
    \item исследование моделей использования RPC в олимпиадной системе BACS,
        а также требований, предъявляемых к RPC в рамках каждой модели;
        % привязать к предметной области. требования из предметной области
    \item исследование существующих технологий RPC и анализ их соответствия
        полученным моделям;
    \item разработка протокола RPC для олимпиадной системы;
    \item реализация протокола для языков программирования Go, Python и C\#.
\end{itemize}

% часть задач должна быть привязана к предметной области

% описание моделей посредством кругов эйлера, как характеристики

\textbf{Объектом исследования} являются распределённые системы.

\textbf{Предметом исследования} являются межпроцессные взаимодействия
в распределённых системах посредством RPC.

\textbf{На защиту выносятся} результаты разработки и исследования моделей RPC,
а также результаты практической реализации технологии RPC.

\textbf{Научная новизна} работы состоит в разработке моделей RPC.

Представленные в диссертации модели описывают качественные особенности
и требования, предъявляемые к RPC. Это вносит ясность в работу инженера
при выборе определённой технологии, предоставляя понятный для него
алгоритм действий.

\textbf{Практическая ценность работы}. На базе полученных моделей
разработана технология RPC, которая позволяет организовать асинхронную
и надёжную передачу данных с учётом распределения нагрузки и горизонтальной
масштабируемости. Данная технология может быть использована для организации
вычислений ряда распределённых систем, в том числе BACS, рассмотренной
в диссертации.

\textbf{Реализация и внедрение результатов работы}. Разработанная технология
внедрена в систему проведения соревнований по спортивному программированию BACS,
разрабатываемую силами студентов ИжГТУ, и является одним из средств обеспечения
отказоустойчивости и масштабирования системы.

%\textbf{Апробация работы.} Результаты работы были представлены на международной
%научно-практической конференции "Молодежь и наука: реальность и будущее".

\textbf{Публикации.} В ходе работы над диссертацией было подано 2 патентные
заявки:
\begin{itemize}
    \item Библиотека программных функций для поддержки сетевого взаимодействия
        между программами в системе <<BACS>>. Принята 10.12.2015.
    \item Программный модуль тестирования корректности работы интернет-сайта
        системы <<BACS>>. Принята 18.05.2016.
\end{itemize}

\textbf{Объём и структура диссертационной работы.} Диссертация содержит
введение, 3 главы и заключение, изложенные на 103 с. машинописного текста,
а также 1 приложения. В работу включены 7 рис., список литературы из 23
наименований. В приложении представлены основные фрагменты исходного кода
реализации разработанного RPC протокола.

анализ межпроцессных взаимодействий
в олимпиадной системе BACS, обзор существующих реализаций,
анализ требований и постановка цели и задач работы

\textbf{В первой главе} анализируются межпроцессные взаимодействия,
делается вывод о целесообразности применения RPC. Делается обзор
существующих реализаций RPC. Обосновывается выбор gRPC в качестве
синхронного RPC для запросов с ограниченным временем работы,
обосновывается необходимость создания асинхронного RPC для запросов
с неограниченным временем работы. Формулируются цели и задачи работы.

\textbf{Во второй главе} подробно рассматриваются связи между удалёнными
компонентами олимпиадной системы BACS и требования к ним. Разрабатывается
протокол асинхронного RPC на основе системы очередей сообщений.

\textbf{В третьей главе} рассматривается процесс разработки RPC на основе
брокера сообщений RabbitMQ. Оценивается производительность реализованного RPC.

\textbf{В заключении} диссертационной работы сформулированы основные выводы
и результаты выполненных исследований и намечены возможные перспективные
направления их развития.

\textbf{В приложении} представлены фрагменты исходного кода реализации
протокола для языков Go, Python и C\#.

% vim: spell spelllang=ru
