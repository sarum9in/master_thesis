\chapter{Исследование моделей использования RPC}
В главе подробно рассматриваются связи между удалёнными компонентами
олимпиадной системы BACS и требования к ним.

\section{Синхронные RPC}
Архив задач является одним из важнейших компонентов олимпиадной системы.
Он содержит все задачи, которые используются для проведения соревнований.
Без задач работа системы не имеет смысла.

Задача представляет из себя набор файлов в определённом формате,
Архив предоставляет унифицированный интерфейс доступа к задачам
вне зависимости от формата. Различным частям олимпиадной системы важны
разные компоненты задачи, в частности Web-интерфейсу необходимо запрашивать
метаинформацию: имя задачи, параметры тестирования, авторов. Подобная
информация имеет ряд характерных особенностей:
\begin{itemize}
    \item размер сильно ограничен, обычно до 1 мегабайта;
    \item присутствует у каждой задачи вне зависимости от формата;
    \item в случае обновления необходимо незамедлительно обновлять
        данные Web-интерфейса, выдавать пользователю устаревшие данные
        недопустимо.
\end{itemize}

\subsection{Интерфейс Архива задач}
Интерфейс Архива задач состоит из двух частей. Первая часть это интерфейс
администратора. Он позволяет добавлять, изменять и скачивать задачи.
Вторая часть это запросы, которые обычно инициируются Web-интерфейсом
системы: получение списка всех задач, подробной и краткой информации
об определённых задачах. В обоих случаях важна производительность каждого
запроса. Если же ответ на запрос не может быть доставлен, то необходимости
повторять попытку нет, так как скорее всего данные устарели, по этой причине
персистентность не является требованием.

\subsection{Обоснование gRPC}
При добавлении задачи в архив передаётся большое количество бинарных данных.
Поэтому их кодирование в текстовых представлениях не является эффективным,
XML-RPC или JSON-RPC применять не следует. Выбор был остановлен на gRPC.
Эта реализация RPC позволяет эффективно организовать передачу данных,
а также их хранение без изменения представления при помощи Google Protocol
Buffers.

\section{Асинхронные RPC}
В отличие от запросов к Архиву задач, при тестировании решения особенности
работы существенно отличаются.

\begin{figure}[H]
    \centering
    \includegraphics[width=\columnwidth]{rs/submit}
    \caption{Решение в системе BACS}
    \label{fig:submit}
\end{figure}

На рисунке~\ref{fig:submit} представлен пример поведения пользователя
и системы при обработке решения. В представленном сценарии пользователь
отправляет в систему решение, которое передаётся Web-интерфейсом
на сервер проверки. Время тестирования решения неизвестно, потому
полезно информировать пользователя о текущем состоянии проверки.
При обновлении страницы пользователь увидит что происходит с его
решением в данный момент времени.

Необходимо иметь ввиду, что пользователей может быть неопределённое количество,
а значит множество решений может находиться на проверке в один момент времени.
Помимо этого, проверка решения может занимать неопределённое время,
обычно до нескольких минут. Выделять на каждого пользователя отдельный
процесс обработки решения в Web-интерфейсе нецелесообразно,
потому применение синхронных RPC не представляется возможным.
Необходима реализация RPC, которая будет контролировать передачу данных
в фоне -- асинхронно, при этом допуская временную недоступность узлов
(Web-интерфейса или сервера проверки), чтобы ресурсоёмкие операции
не приходилось повторять заново, то есть имела свойство персистентности.

В ходе анализа показано, что ни одна из рассмотренных реализаций RPC
не решает проблему персистентности. В данной работе рассматривается
реализация асинхронного персистентного RPC протокола.
