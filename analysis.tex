\chapter{Аналитический обзор межпроцессных взаимодействий
и формирование требований к ним}

Целью данной главы является анализ межпроцессных взаимодействий
в олимпиадной системе BACS, обзор существующих реализаций,
анализ требований и постановка цели и задач работы.

\section{Олимпиадная система BACS}
Олимпиадная система BACS это распределённая система проведения соревнований
и лабораторных работ среди учащихся по дисциплинам, связанным
с программированием. Система состоит из нескольких компонент,
связанных между собой. Есть два типа связи (см.~рис.~\ref{fig:bacsbasicrpc}):
\begin{itemize}
    \item Синхронные связи: система не может продолжать работу
        без получения ответа. Web-интерфейс и Архив связаны именно таким
        образом.
    \item Асинхронные связи: для работы системы получение ответа не обязательно
        должно происходить сразу. К примеру тестирование пользовательского
        решения какой-либо задачи может занимать продолжительное время.
\end{itemize}

\begin{figure}[H]
    \centering
    \input rs/bacsbasicrpc.dot
    \caption{Связи между компонентами BACS}
    \label{fig:bacsbasicrpc}
\end{figure}

Такие связи являются примером межпроцессного взаимодействия.

\section{Межпроцессное взаимодействие}
Межпроцессное взаимодействие IPC (inter-process communication) --
это обмен данными между отдельными процессами.
Процессы могут быть запущены как в одном адресном пространстве,
так и на удалённых компьютерах. Обеспечиваются такие взаимодействия
как посредством ядра операционной системы, так и при помощи механизмов
пользовательского пространства (к примеру внешних программных модулей).

Среди механизмов IPC выделяют механизмы обмена сообщениями, синхронизации,
разделения памяти и удалённых вызовов (RPC -- remote procedure call).
Для распределённых систем актуальны механизмы, независимые от ядра операционной
системы и позволяющие реализовывать связь между узлами, запущенными
под управлением различных операционных систем. К таким относятся механизмы
обмена сообщениями и удалённых вызовов.

\subsection{Обмен сообщениями}
Обмен сообщениями -- это форма связи, используемая в параллельных вычислениях,
реализуемая путём посылки пакетов информации получателям, которые могут
содержать команды и уведомления, а также данные для обработки.

Обычно обмен сообщениями реализуется асинхронно, сообщение может быть доставлено
неопределённому кругу получателей, в том числе независимо от отправителя,
если используется программа-посредник -- брокер. Наличие брокера с одной стороны
приводит к централизации системы, с другой позволяет упростить архитектуру.
Брокер хранит сообщения до момента доставки, уведомляет об ошибках,
реализует маршрутизацию сообщений.

\subsection{RPC}
Удалённый вызов процедур -- это класс технологий, позволяющих производить
вызов процедур одной компьютерной программы из адресного пространства другой.
Как правило реализации RPC позволяют абстрагироваться от деталей сетевого
взаимодействия, фактически стирая разницу между вызовом удалённой и локальной
процедуры.

Технологии RPC удобно применять для организации вычислений в распределённых
системах. В языках программирования высокого уровня команда есть вызов процедуры.
В распределённых системах таким образом команда есть удалённый вызов процедуры.
RPC позволяет передавать структурированные данные небольшого объёма (десятки
мегабайт). Так как используются типы данных языков программирования высокого
уровня, происходит прозрачная интеграция различных частей системы.

Любая технология RPC состоит из двух частей: протокола RPC и
реализации протокола для конкретных языков программирования. Следует обратить
внимание, что для большинства языков программирования следует создавать
отдельную реализацию (в некоторых случаях обёртку к существующей),
так как в каждом языке есть свои характерные особенности стиля работы,
что позволяет сделать реализацию максимально приближенной к языку и понятной
пользователям.

Протокол RPC с другой стороны является общим, он един для всех реализаций.
Это позволяет объединять различные узлы распределённой системы, реализованные
на различных языках программирования.

Протокол RPC обычно является двухуровневым.
Уровень представления данных определяет методы кодирования информации,
такие как байтовое представление чисел и других простых типов данных,
массивов, строк, структур.
Транспортный уровень определяет механизмы передачи уже закодированных байт
по сети. Часто используется какой-либо существующий протокол в качестве
транспортного.

\section{Требования к технологии RPC}
Среди требований, предъявляемых к RPC, выделяют:
\begin{itemize}
    \item переносимость данных;
    \item надёжность;
    \item производительность;
    \item персистентность;
    \item диагностика неисправностей.
\end{itemize}

\subsection{Переносимость данных}
Разные части системы могут быть запущенны на различных платформах.
Это включает в себя как различные архитектуры процессора, операционные системы,
так и языки программирования и интерпретаторы. Представление данных на различных
платформах может существенно отличаться, к примеру, различным порядком байт,
выравниванием, размером указателя, представлением строк и массивов. Простое
копирование представления данных из оперативной памяти не позволит
взаимодействовать различающимся частям системы. Необходим механизм сериализации
в платформо-независимое представление для последующей передачи в другую часть
системы.

Для системы BACS этот критерий является обязательным, так как одним из основных
качеств системы является модульность. Это означает заменяемость компонент
без изменения интерфейса, а значит возможность замены к примеру языка
программирования в отдельных компонентах.

\subsection{Надёжность и производительность}
Отдельные компоненты могут быть запущены на различных серверах. Сбои при работе
сети, технические работы на сервере (даже кратковременные), моменты пиковой
нагрузки с временным отказом от обработки новых запросов, – всё это может
приводить к недоступности или медленной работе отдельных узлов системы.
При запросах к этим узлам остальные узлы должны адекватно реагировать на их
недоступность, корректно определяя это в приемлемое время, повторяя запрос
при необходимости.

Существует два типа запросов:
\begin{itemize}
    \item Запрос существующих данных, пример запросов между web-интерфейсом и
        архивом задач приведён на рисунке~\ref{fig:bacsbasicrpc}.
        При штатной работе системы обработка запроса занимает строго
        ограниченное время. В случае сбоя запрос может зависнуть.
        Необходимо прекращать обработку данного запроса,
        информируя вызывающую сторону об ошибке.
    \item Запрос на совершение ресурсоёмкой операции. Такие запросы происходят
        между web интерфейсом и сервером тестирования решений.
        Длительность тестирования решения заранее неизвестна, потому следует
        обеспечить надёжность асинхронной обработки запроса. При этом ответ
        web-интерфейсу отправляется сервером тестирования самостоятельно.
\end{itemize}

\subsection{Персистентность}
В асинхронных связях данные являются более ценными, чем в синхронных.
Это связано со стоимостью их получения: обработка асинхронного запроса
может занимать существенно больше времени, потому желательно избегать
повторения запросов без крайней на то необходимости. Эта проблема решается
при помощи персистентного хранения запросов и ответов, что даёт возможность
получения ответа на запрос, даже если обработчик уже прекратил свою работу,
или отправки запроса при отсутствии свободных обработчиков.

Ни одна из рассмотренных RPC технологий не реализует это требование.
Для его реализации необходим промежуточный сервис -- брокер.

\subsection{Диагностика неисправностей}
При обработке запросов могут возникать ошибки. Важно как можно более полно
передавать информацию об этих ошибках для последующей диагностики системными
администраторами или разработчиками системы. В информацию об ошибке может
входить стек вызовов, время вызова, данные запроса, идентификаторы вызывающей
и принимающей сторон.

\section{Обзор существующих технологий RPC}
\subsection{XML-RPC}
Использует XML для представления данных, HTTP в качестве транспортного протокола.

\noindent Пример запроса:
\begin{verbatim}
<?xml version="1.0"?>
<methodCall>
  <methodName>examples.getStateName</methodName>
  <params>
    <param>
     <value><i4>41</i4></value>
    </param>
  </params>
</methodCall>
\end{verbatim}

\noindent Пример ответа:
\begin{verbatim}
<?xml version="1.0"?>
<methodResponse>
  <params>
    <param>
      <value><string>South Dakota</string></value>
    </param>
  </params>
</methodResponse>
\end{verbatim}

Сильными сторонами технологии являются исключительная простота
и распространённость среди языков программирования. Недостатками являются
низкая производительность и избыточность представления данных (недостаток XML).
Последнее влечёт за собой чрезмерную нагрузку на сеть при значительных объёмах
передаваемых данных. В случае же если преобладают бинарные данные,
они кодируются в base64, специальной кодировке для представления бинарных
данных в текстовом виде. Особенностью данной кодировки является увеличение
размера передаваемой информации на 33\%~\cite{base64rfc}.

\subsection{JSON-RPC}
Технология аналогична XML-RPC, за исключением использования JSON вместо XML.
Удобна для реализации в Web ввиду распространённости поддержки JSON
в браузерах. Сильные и слабые стороны те же.

\subsection{D-BUS}
Система для организации RPC в пределах одной операционной системы.
Оперирует концепцией сервиса: каждое серверное приложение при запуске
регистрирует сервис с заранее известным именем. Клиенты будут обращаться
к сервису при помощи данного имени.

Сообщения в D-Bus бывают четырёх видов: вызовы методов, результаты вызовов методов, сигналы (широковещательные сообщения) и ошибки.

Используется собственный бинарный протокол представления данных.
В качестве транспортного протокола используется как правило доменный сокет UNIX,
но может быть применён и TCP.

Сильными сторонами являются высокая скорость работы и широкая поддержка
в языках программирования. Так как технология ориентирована на работу
в пределах одного компьютера, для распределённых систем не подходит.

\subsection{Java RMI}
Java remote method invocation -- программный интерфейс вызова удалённых методов
в языке Java.
\begin{verbatim}
import java.rmi.Remote;
import java.rmi.RemoteException;

public interface RmiServerIntf extends Remote {
    public String getMessage() throws RemoteException;
}
\end{verbatim}

Для кодирования данных используется встроенная в язык технология сериализации.
В качестве транспортного протокола используется TCP с реализованным
поверх него собственным бинарным протоколом.

Сильными сторонами являются высокая скорость работы и бесшовная интеграция
с кодом на языке Java. Недостатком -- поддержка только платформы Java.
Последнее не позволяет создавать по-настоящему кросс-платформенные интерфейсы
в распределённых приложениях, если только для разработки не используется
исключительно Java.

\subsection{SOAP}
Simple Object Access Protocol -- простой протокол доступа к объектам.
Является расширением XML-RPC. Он является более гибким, но также добавляет
и свои недостатки к XML-RPC: существуют несовместимые реализации протокола.
Технология, как и XML-RPC, снижает скорость работы за счёт передачи данных
в форме XML.

\subsection{Apache Thrift}
Это язык описания интерфейсов для определения и создания служб
на различных языках программирования. Является фреймворком RPC.
Содержит генератор кода для таких языков как C\#, C++, Cappuccino, Cocoa,
Delphi, Erlang, Go, Haskell, Java, OCaml, Perl, PHP, Python, Ruby и Smalltalk.

Поддерживает множество форматов передачи и кодирования данных, среди них
бинарные и текстовые, в том числе отладочный (легко читаемый человеком), JSON.

Поддерживает различные транспортные протоколы, в том числе сетевые.

Сильными сторонами являются гибкость, распространённость реализаций
для различных языков программирования и совместимость между ними, высокая
скорость работы.
Основным недостатком выделяют низкое качество документации.
Кроме этого технология сериализации данных неотделима от фреймворка,
потому её использование для сохранения данных например на жёсткий диск
может быть затруднительно.

Технология широко применяется компанией Facebook для построения распределённых
систем.

\subsection{gRPC}
gRPC -- высокопроизводительный фреймворк для построения сервисов и клиентов.
Разрабатывается компанией Google, находится в открытом доступе.

Технология использует Google Protocol Buffers в качестве технологии
представления данных. Эта технология широко распространена среди
различных языков программирования.

В качестве протокола передачи данных используется HTTP/2 -- современная
версия протокола HTTP. Она отличается поддержкой параллельной передачи
данных в рамках одного соединения.

gRPC имеет реализации для C, C++, Java, Go, Node.js, Python, Ruby, Objective-C,
PHP и C\#. Отличается высокой скоростью работы и совместимостью между
реализациями. К недостаткам относят сложность анализа закодированных данных.
Также следует иметь ввиду что в настоящее время многие реализации являются
новыми и не получили достаточной стабильности работы.

\section{Постановка цели и задач работы}
Среди рассмотренных технологий RPC лучше всего удовлетворяют требованиям
синхронных связей олимпиадной системы gRPC и Apache Thrift. Обе технологии
имеют реализацию для большинства популярных платформ и языков программирования.
Предпочтение отдано gRPC по причине лучшей модульности: возможность
использовать одинаковый формат данных для хранения и передачи. gRPC
использует Google Protocol Buffers, который удобно применять при сохранении
данных на жёсткий диск.

Существует второй тип связи -- асинхронный. Для асинхронной связи
важна персистентность, то есть возможность продолжить обработку сообщения
даже в случае временной недоступности отправителя или получателя.
Эта возможность обычно реализуется при помощи промежуточного сервера --
брокера сообщений. Ни одна из рассмотренных технологий не удовлетворяет
требованию персистентности, таким образом необходимо разработать данную
технологию.

Разработка ведётся в два этапа:
\begin{enumerate}
    \item разработка протокола, независимого от платформы (операционной системы
        или языка программирования);
    \item разработка клиентов для различных платформ, для олимпиадной системы
        BACS важны Go, Python и C\#.
\end{enumerate}

\section{Выводы}
В олимпиадной системе BACS есть различные виды связей между
удалёнными компонентами. Для реализации синхронных связей применимы
существующие технологии, выбор остановлен на gRPC. В случае же с асинхронной
связью показано, что нет подходящей реализации RPC, потому ставится задача
разработки асинхронной технологии RPC на основе брокера сообщений.
