\chapter{Аналитический обзор межпроцессных взаимодействий
и формирование требований к ним}

Целью данной главы является анализ межпроцессных взаимодействий
в олимпиадной системе BACS, обзор существующих реализаций,
анализ требований и постановка цели и задач работы.

\section{Общая характеристика}
Межпроцессное взаимодействие IPC (inter-process communication) --
это обмен данными между отдельными процессами.
Процессы могут быть запущены как в одном адресном пространстве,
так и на удалённых компьютерах. Обеспечиваются такие взаимодействия
как посредством ядра операционной системы, так и при помощи механизмов
пользовательского пространства (к примеру внешних программных модулей).

Среди механизмов IPC выделяют механизмы обмена сообщениями, синхронизации,
разделения памяти и удалённых вызовов (RPC -- remote procedure call).
Для распределённых систем актуальны механизмы, независимые от ядра операционной
системы и позволяющие реализовывать связь между узлами, запущенными
под управлением различных операционных систем. К таким относятся механизмы
обмена сообщениями и удалённых вызовов.

\subsection{Обмен сообщениями}
Обмен сообщениями -- это форма связи, используемая в параллельных вычислениях,
реализуемая путём посылки пакетов информации получателям, которые могут
содержать команды и уведомления, а также данные для обработки.

Обычно обмен сообщениями реализуется асинхронно, сообщение может быть доставлено
неопределённому кругу получателей, в том числе независимо от отправителя,
если используется программа-посредник -- брокер. Наличие брокера с одной стороны
приводит к централизации системы, с другой позволяет упростить архитектуру.
Брокер хранит сообщения до момента доставки, уведомляет об ошибках,
реализует маршрутизацию сообщений.

\subsection{RPC}
Удалённый вызов процедур -- это класс технологий, позволяющих производить
вызов процедур одной компьютерной программы из адресного пространства другой.
Как правило реализации RPC позволяют абстрагироваться от деталей сетевого
взаимодействия, фактически стирая разницу между вызовом удалённой и локальной
процедуры.

Технологии RPC удобно применять для организации вычислений в распределённых
системах. В языках программирования высокого уровня команда есть вызов процедуры.
В распределённых системах таким образом команда есть удалённый вызов процедуры.
RPC позволяет передавать структурированные данные небольшого объёма (десятки
мегабайт). Так как используются типы данных языков программирования высокого
уровня, происходит прозрачная интеграция различных частей системы.

\section{Обзор существующих технологий RPC}
\subsection{XML-RPC}
\subsection{JSON-RPC}
\subsection{D-BUS}
\subsection{Java RMI}
\subsection{SOAP}
\subsection{Apache Thrift}
\subsection{gRPC}

\section{Требования к технологии RPC}

\section{Выводы, постановка цели и задач работы}
