\documentclass[xetex,mathserif,serif,10pt]{beamer}
%\documentclass[11pt]{article}
\usepackage{xltxtra}
\usepackage{polyglossia}
\setdefaultlanguage[spelling=modern]{russian}
%\setmainfont[Mapping=tex-text]{DejaVu Sans}
%\setmainfont[Mapping=tex-text]{Liberation Sans}
\setmonofont[Mapping=tex-text]{DejaVu Sans Mono}
\setmainfont[Mapping=tex-text]{Linux Libertine O}
%\setmonofont[Mapping=tex-text]{Liberation Mono}

\input dot2tex

\usepackage{verbatim}
\usepackage{tabularx}
\usepackage{float}
\usepackage{url}

\usepackage{textpos}

\usepackage{hyperref}

\usepackage{indentfirst}

\usepackage{algorithm}
\usepackage{algorithmic}

\usepackage{graphicx}

\usepackage{dirtree}

\usepackage{listings}
\lstset{ %
language=C++,                   % the language of the code
basicstyle=\ttfamily\footnotesize,% the size of the fonts that are used for the code
numbers=left,                   % where to put the line-numbers
numberstyle=\footnotesize,      % the size of the fonts that are used for the line-numbers
stepnumber=2,                   % the step between two line-numbers. If it's 1, each line
                                % will be numbered
numbersep=5pt,                  % how far the line-numbers are from the code
%backgroundcolor=\color{white},  % choose the background color. You must add \usepackage{color}
showspaces=false,               % show spaces adding particular underscores
showstringspaces=false,         % underline spaces within strings
showtabs=false,                 % show tabs within strings adding particular underscores
frame=single,                   % adds a frame around the code
tabsize=2,                      % sets default tabsize to 2 spaces
captionpos=b,                   % sets the caption-position to bottom
breaklines=true,                % sets automatic line breaking
breakatwhitespace=false,        % sets if automatic breaks should only happen at whitespace
%title=\lstname,                 % show the filename of files included with \lstinputlisting;
                                % also try caption instead of title
escapeinside={\%*}{*)},         % if you want to add a comment within your code
morekeywords={*,...}            % if you want to add more keywords to the set
}

%\setbeamertemplate{caption}[numbered]
\setbeamertemplate{footline}[frame number]

\input config

%\usetheme{Warsaw}
\usetheme{Singapore}

\newenvironment{sframe}[2]{\section{#1}\begin{frame}[label=#2]{#1}}{\end{frame}}

\begin{document}
    \title[RPC]{\titletext}
    \author[Филиппов]{А.~Филиппов\inst{1}}
    \institute
    {
        \inst{1}
        Ижевский государственный технический университет имени М.Т.~Калашникова
    }

    \input beamertitle.tex

    \begin{sframe}{Цель работы}{target}
        \begin{block}{Цель}
            Анализ и разработка технологии RPC для олимпиадной системы.
        \end{block}

        \begin{block}{Требования}
            \begin{itemize}
                \item Надёжность передачи данных
                \item Кросс-платформенность
            \end{itemize}
        \end{block}
    \end{sframe}

    \begin{sframe}{Задачи работы}{problems}
        \begin{itemize}
            \item \hyperlink{analysis}{Исследование моделей использования RPC
                в олимпиадной системе BACS, а также требований,
                предъявляемых к RPC в рамках каждой модели};
            \item \hyperlink{review}{исследование существующих технологий RPC
                и анализ их соответствия полученным моделям;}
            \item \hyperlink{protodev}{разработка протокола RPC
                для олимпиадной системы;}
            \item \hyperlink{binddev}{реализация протокола для языков
                программирования Go, Python и C\#.}
        \end{itemize}
    \end{sframe}

    \begin{sframe}{Исследование моделей использования RPC}{analysis}
        \begin{itemize}
            \item Асинхронность
            \item Надёжность
            \item Переносимость
            \item Производительность
        \end{itemize}
        \begin{figure}
            \centering
            \resizebox*{!}{0.6\textheight}{\input rs/bacsbasicrpc.dot}
        \end{figure}
    \end{sframe}

    \begin{sframe}{Исследование технологий RPC}{review}
        \begin{center}
            \resizebox{\columnwidth}{!}{\begin{tabular}{|c|c|c|c|c|} \hline
                                & Асинхронность & Персистентность &
                                    Переносимость & Производительность \\ \hline
                XML-RPC         & + & - & + & низкая \\ \hline
                JSON-RPC        & + & - & + & низкая \\ \hline
                D-BUS           & + & - & - & высокая \\ \hline
                Java RMI        & + & - & - & высокая \\ \hline
                SOAP            & + & - & + & низкая \\ \hline
                Apache Thrift   & + & - & + & высокая \\ \hline
                gRPC            & + & - & + & высокая \\ \hline
            \end{tabular}}
        \end{center}
    \end{sframe}

    \begin{sframe}{Разработка протокола}{protodev}
        \begin{figure}
            \centering
            \resizebox{\columnwidth}{!}{\input rs/brokerproto.dot}
        \end{figure}
    \end{sframe}

    \begin{sframe}{Реализация протокола}{binddev}
        \begin{block}{Языки}
            \begin{itemize}
                \item Go: github.com/streadway/amqp
                \item Python: python-pika
                \item C\#: RabbitMQ.Client
            \end{itemize}
        \end{block}
        \begin{block}{Отправитель}
            \begin{itemize}
                \item Подключение и мониторинг соединения с брокером
                \item Уведомления пользователя о результатах и промежуточных
                    состояниях
            \end{itemize}
        \end{block}
        \begin{block}{Получатель}
            \begin{itemize}
                \item Асинхронная отправка промежуточных состояний
                \item Последовательная обработка заданий
            \end{itemize}
        \end{block}
    \end{sframe}

    \begin{frame}
        \Large\centering Спасибо за внимание!
    \end{frame}

    \begin{frame}{Содержание}
        \tableofcontents
    \end{frame}
\end{document}
